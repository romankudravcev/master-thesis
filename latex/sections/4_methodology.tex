%
\section{Methodology Approach}
\label{sec:methodology-approach}
%

%TODO GRAFIK MIT CONTRIBUTIONS VERBUNDEN MIT DEN KAPITELN UND DEM BACKGROUND

This thesis adopts an empirical-analytical approach, combining experimental benchmarking, comparative analysis, and design science research (DSR) to develop and evaluate a migration framework for applications in edge orchestration environments. This results in the use of a mixed methods approach, combining qualitative and quantitative methods. 

In the qualitative part, we will mainly focus on identifying and evaluating existing tools and methodologies to address specific migration challenges, such as state management and secure networking. To answer research questions 1 and 2, a systematic review of the literature and available technologies will be conducted, their capabilities will be analyzed, and their suitability for the migration context will be evaluated. The tools will be selected not only based on the suitability, but also on the basis of some selection criteria that the tools have to fulfill: open source, community support, scalability, compatibility with different infrastructures. Based on these criteria, these tools will be integrated into the migration framework.

The quantitative part involves benchmarking and experimental evaluation to test and compare an orchestrated environment before and during migration. Objective metrics, such as availability, response time, CPU utilization, and memory consumption, will be measured. Additionally, the duration of the migration process will be recorded to compare the efficiency of different tools. This analysis is critical to assess the additional load that the migration process may place on the system beyond the resource requirements of the applications or services being migrated. The results will facilitate research question 3.

Following the benchmarking and systematic review, the tools will be re-evaluated to determine their overall suitability. This iterative evaluation will help identify the tools best suited to ensuring high efficiency and service availability in the migration process, thereby answering Research Question 4.

Once the qualitative and quantitative analyses are complete, and suitable tools have been identified - or new tools have been developed as needed - the DSR phase will beginn. This phase involves finalizing the development of the migration application. Initial development work will have been undertaken earlier to evaluate tool suitability during benchmarking, but the implementation will be refined and completed at this stage.